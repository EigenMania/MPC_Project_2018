\documentclass[12pt]{article}

\usepackage{graphicx}
\usepackage{soul}
\usepackage{xcolor}
\usepackage[margin=0.75in]{geometry}

\begin{document}

%% Title Page
\thispagestyle{empty}

\begin{center}
    \LARGE \textbf{2018 MPC Programming Exercise} \\
    \Large \textbf{Quadrotor Control}
    \normalsize
\end{center}

\vfill

\begin{center}
    \includegraphics[width=0.9\textwidth]{Figures/quadcopter_comic.png}
\end{center}

\vfill

\begin{center}
    \begin{tabular}{l  l  l }
        \textbf{Name} & \textbf{Legi \#} & \textbf{Email} \\
        Adrian Esser & 17-937-954 & aesser@student.ethz.ch \\
        Thomas Lew & \hl{...} & \hl{...} \\
    \end{tabular}
\end{center}

\clearpage
\newpage

\tableofcontents

\clearpage
\newpage

\section{Q1 - Interpretation of Linearized System}

\begin{center}
    \textit{Interpretation of the structure of matrices $\mathrm{A}^c$ and $\mathrm{B}^c$.
    Explain in particular the nonzero rows 4 and 5 of $\mathrm{A}^c$ and the nonzero rows of
    $\mathrm{B}^c$ in connection with the nonlinear dynamics described above.} \textbf{2.5\%}
\end{center}
    

\section{Q2 - Choice of Tuning Parameters}

\begin{center}
    \textit{Choice of tuning parameters ($\mathrm{Q}, \mathrm{R}, \mathrm{P},
    \mathrm{A}_{\mathcal{X}_{f}}, \mathrm{b}_{\mathcal{X}_{f}}$) and motivation for them.} \textbf{5\%}
\end{center}
 
\section{Q3 - Initial Reponse Plots}

\begin{center}
    \textit{Plots of the response starting from the given initial condition $x^1(0)$.} \textbf{10\%}
\end{center}
 
\section{Q4 - Steady State Reference Tracking}

\begin{center}
    \textit{Define the steady state ($x_r, u_r$) as a function of n arbitrary reference $r^1$.} \textbf{5\%}
\end{center}
 
\section{Q5 - Reference Signal Plots}

\begin{center}
    \textit{Plots of the response or the constant reference signal $r^1 = \left[1.0000 \ 0.1745 \
    -0.1745 \ 1.7453 \right]^\mathrm{T}$}. \textbf{10\%}
\end{center}
 
\section{Q6 - Varying Reference Signal Plots}

\begin{center}
    \textit{Plots of the response for the slowly varying reference signal $r^1(k) = \left[1 \
    0.1745 \cdot \mathrm{sin}(T_s k) \ -0.1745 \cdot \mathrm{sin}(T_s k) \ \pi/2 \right]^\mathrm{T}$.} \textbf{2.5\%}
\end{center}

\section{Q7 - Nonlinear Model Reference Tracking}

\begin{center}
    \textit{Plots of a reference tracking response of the nonlinear model.} \textbf{5\%}
\end{center}
 
\section{Q8 - Disturbance Observer Design}

\begin{center}
    \textit{Provide the matric L and justify your choice.} \textbf{5\%}
\end{center}
 
\section{Q9 - Reference Signal Plots}

\begin{center}
    \textit{Plots of the response or the constant reference signal $r^1 = \left[0.8 \ 0.12 \
    -0.12 \ \pi/2 \right]^\mathrm{T}$}. \textbf{10\%}
\end{center}
 
\section{Q10 - Varying Reference Signal Plots}

\begin{center}
    \textit{Plots of the response for the slowly varying reference signal $r^1(k) = \left[0.8 \
    0.12 \cdot \mathrm{sin}(T_s k) \ -0.12 \cdot \mathrm{sin}(T_s k) \ \pi/2 \right]^\mathrm{T}$.} \textbf{2.5\%}
\end{center}


\end{document}
